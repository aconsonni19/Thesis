\documentclass[12pt, a4paper, oneside]{report}
\usepackage[paper=a4paper,margin=1in]{geometry}
\usepackage{csquotes}
\usepackage[provide = *, italian]{babel}
\usepackage{graphicx} % allows inserting graphics
\usepackage{listings}
\usepackage{algpseudocode} % allows code snippets
\usepackage{algorithm}
\usepackage[hidelinks]{hyperref} % allows
\usepackage{subcaption} 
\usepackage[export]{adjustbox} 
\usepackage{wrapfig} % allows warped figures
\usepackage[acronym]{glossaries} % add glossary
\usepackage{acronym} % add acronyms
\usepackage{subfiles} % support multiple files
\usepackage{fancyhdr} % more control on headers and footers
\usepackage{xcolor}
\usepackage[
backend=biber,
style=ieee,
sorting=none,
]{biblatex}
\usepackage{ragged2e}
\usepackage{fancyhdr}
\usepackage{tikz}
\usepackage[normalem]{ulem}
\usepackage{array}
\usepackage{tabularx}
\usepackage[export]{adjustbox}


\useunder{\uline}{\ul}{}
\usetikzlibrary{shapes.geometric, arrows.meta, positioning, fit, patterns}
\usetikzlibrary{calc}
\pagestyle{fancy} %oppure none
\fancyhf{}
\fancyhead[L]{\textit{Capitolo \thechapter}}
\fancyfoot{}
\fancyhead[R]{\textit{Pagina \thepage}}

\graphicspath{ {./images/} }

\addbibresource{./bibfile.bib}

%\includeonly{chapters/1-Introduzione}

\definecolor{codegreen}{rgb}{0,0.6,0}
\definecolor{codegray}{rgb}{0.5,0.5,0.5}
\definecolor{codepurple}{rgb}{0.58,0,0.82}
\definecolor{backcolour}{rgb}{0.95,0.95,0.92}

\lstdefinestyle{mystyle}{
    backgroundcolor=\color{backcolour},   
    commentstyle=\color{codegreen},
    keywordstyle=\color{magenta},
    numberstyle=\tiny\color{codegray},
    stringstyle=\color{codepurple},
    basicstyle=\ttfamily\footnotesize,
    breakatwhitespace=false,         
    breaklines=true,                 
    captionpos=b,                    
    keepspaces=true,                 
    numbers=left,                    
    numbersep=5pt,                  
    showspaces=false,                
    showstringspaces=false,
    showtabs=false,                  
    tabsize=2
}

\lstset{style=mystyle}



\begin{document}
    \subfile{frontispiece.tex}
    \thispagestyle{empty}
    \vspace*{\fill}
    \vspace*{-4.5cm}
    \begin{Center}
        \begin{minipage}{\textwidth}
            \centering
            \itshape
            Vorrei esprimere i miei più sentiti ringraziamenti al \textbf{prof. Claudio Ferretti}, alla \textbf{Dott.sa Martina Saletta} e al \textbf{prof. Giovanni Denaro} per il costante supporto, la disponibilità e i preziosi consigli offerti durante tutto il percorso di tesi.
            \newline \vspace*{1.5em}
            Un ringraziamento speciale va alla \textbf{mia famiglia}, che con il suo supporto incondizionato mi ha permesso di affrontare con serenità questo percorso di studi. \newline \vspace*{1.5em}
            Sono profondamente grato anche \textbf{a tutti i miei amici}, il quale costante sostegno mi ha accompagnato e incoraggiato lungo tutto il percorso universitario
        \end{minipage}
    \end{Center}
    \vspace*{\fill}
    \newpage
    \clearpage
    % Abstract
    \thispagestyle{empty}
    \vspace*{\fill}
    \vspace*{-4cm}
    \begin{Center}
        \begin{minipage}{\textwidth}
            \centering
            \textbf{Abstract} \newline
            La sicurezza informatica è sempre più centrale nello sviluppo software, soprattutto quando si analizzano applicazioni senza accesso al codice sorgente. Questa tesi presenta Binoculars, una piattaforma web pensata per semplificare l’analisi di sicurezza di file binari ELF per architettura x86. Basata su angr, Flask e SvelteKit, la piattaforma offre un’interfaccia intuitiva che consente anche ad analisti non specialisti di effettuare una prima valutazione automatizzata, facilitando l’identificazione di potenziali vulnerabilità.
        \end{minipage}
    \end{Center}
    \vspace*{\fill}
    \addtocounter{page}{-2}

    \tableofcontents

    \thispagestyle{empty}
    \addtocounter{page}{-1}
    
    \listoffigures

    \lstlistoflistings

    \listoftables

    \thispagestyle{empty}
    \addtocounter{page}{-3}



    % Capitolo 1
    \subfile{chapters/1-Introduzione.tex}
    % Capitolo 2
    \subfile{chapters/2-stato_dell_arte.tex}
    % Capitolo 3
    \subfile{chapters/3-metodologia.tex}
    % Capitolo 4
    \subfile{chapters/4-tecnologie.tex}
    % Capitolo 5
    \subfile{chapters/5-analisi.tex}
    % Capitolo 6
    \subfile{chapters/6-implementazione.tex}


    \clearpage
    \fancypagestyle{bibliopage}{
        \fancyhf{}
        \fancyhead[R]{\textit{Pagina \thepage}}
        \fancyhead[L]{\textit{Bibliografia}} % oppure lascia vuoto se preferisci
    }

    \pagestyle{bibliopage}
    \printbibliography[heading=bibintoc]
\end{document}