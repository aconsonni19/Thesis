\documentclass[../main.tex]{subfiles}
\addbibresource{../bibfile.bib}

\begin{document}
\chapter{Sperimentazione}
In questo capitolo, verranno presentati e analizzati i risultati ottenuti dall'applicazione
delle strategie di analisi disponibili sulla piattaforma (\textit{Arbiter} e \textit{Binoculars}).
Per ogni classe di vulnerabilità considerata, sono stati considerati sia programmi vulnerabili
che non vulnerabili.
\section{Sperimentazioni con VulnDetect}
Come già esposto nel \textit{capitolo 5}, la strategia di analisi implementata da \textit{VulnDetect} è capace
solamente di rilevare vulnerabilità di tipo \textit{stack-based buffer overflow} (CWE-121). L'efficacia di questa
strategia è stata verificata tramite il test di tre file binari.
\subsection{Test 1: ahgets1-bad}
Questo programma proviene da una delle test suite messe a disposizione dal \textit{Software Assurance Reference Dataset} (SARD).
Il codice del programma è il seguente:
\lstinputlisting[language = C, caption = Codice del programma \textit{ahgets1-bad}]{../tests/VulnDetect/ahgets1-bad.c}
Durante il ciclo di scrittura, il programma non controlla mai se i dati hanno superato il limite del buffer. Il programma è quindi
vulnerabile a \textit{stack-based buffer overflow}. I risultati presentati da \textit{VulnDetect} confermano la presenza di una vulnerabilità di questo tipo:
    \begin{table}[H]
        \centering
        \begin{tabular}{|l|l|l|}
        \hline
        Address  & Type                                 & Description                                                                                                  \\ \hline
        0x40118b & CWE-121: Stack-based buffer overflow  & \begin{tabular}[c]{@{}l@{}}A potential instruction pointer \\ hijack by user input was detected\end{tabular} \\ \hline
        0x40117a & CWE-121: Stack-based buffer overflow & \begin{tabular}[c]{@{}l@{}}A potential instruction pointer \\ hijack by user input was detected\end{tabular} \\ \hline
        \end{tabular}
        \caption{Vulnerabilità rilevate da \textit{VulnDetect} sul programma ahgets1-bad}
        \end{table}
\subsection{Test 2: gets1-bad}
Come il precedente, anche questo programma proviene da una delle test suite messe a disposizione dal \textit{SARD}.
Il codice del programma è il seguente:
\lstinputlisting[language = C, caption = Codice del programma \textit{gets1-bad}]{../tests/VulnDetect/gets1-bad.c}
La mancanza di controlli sulla lunghezza dei dati che la funzione \textit{fgets} scrive nel buffer rende questo programma vulnerabile
a \textit{stack-based buffer overflow}. Anche in questo caso, i risultati presentati da \textit{VulnDetect} confermano la presenza della vulnerabilità:
\begin{table}[H]
    \centering
    \begin{tabular}{|l|l|l|}
    \hline
    Address  & Type                                 & Description                                                                                                  \\ \hline
    0x401159 & CWE-121: Stack-based buffer overflow  & \begin{tabular}[c]{@{}l@{}}A potential instruction pointer \\ hijack by user input was detected\end{tabular} \\ \hline
    0x40118d & CWE-121: Stack-based buffer overflow & \begin{tabular}[c]{@{}l@{}}A potential instruction pointer \\ hijack by user input was detected\end{tabular} \\ \hline
    \end{tabular}
    \caption{Vulnerabilità rilevate da \textit{VulnDetect} sul programma gets1-bad}
\end{table}
\newpage
\subsection{Test 3: gets1-good}
Questo programma contiene una versione corretta e non vulnerabile del codice presentato per il programma \textit{gets1-bad}.
Il codice del programma è quindi il seguente:
\lstinputlisting[language = C, caption = Codice del programma \textit{gets1-good}]{../tests/VulnDetect/gets1-good.c}
Il risultato ottenuto dell'analisi del programma con \textit{VulnDetect} è anche in questo caso quello atteso: il programma non risulta vulnerabile
a \textit{stack-based buffer overflow}.
\section{Sperimentazioni con Arbiter}
\textit{Arbiter} è capace di rilevare quattro classi di vulnerabilità:
\begin{itemize}
    \item \textbf{CWE-131: Incorrect calculation of buffer size}: Occorre quando un programma alloca un buffer di dimensione più
    piccola di quella richiesta. Questo tipo di vulnerabilità può portare quindi a \textit{buffer overflow}.
    \item \textbf{CWE-134: Controlled format string}: Occorre quando un programma utilizza in maniera impropria le funzioni della famiglia
    \textit{printf}. La presenza di questo tipo di vulnerabilità permette all'attaccante di utilizzare particolari \textit{format specifiers} (come \textit{\%x} e \textit{\%n})
    per leggere o modificare i valori presenti in memoria.
    \item \textbf{CWE-252: Unchecked return value}: Un errore comune da parte di un programmatore è l'assumere che l'esecuzione di una certa funzione vada sempre a buon fine oppure che ritorni sempre
    un valore tra quelli previsti. Tuttavia, se un attaccante riesce a far fallire l'esecuzione della funzione, allora il programma potrebbe trovarsi in uno stato diverso da quello assunto
    dal programmatore e quindi potrebbe risultare vulnerabile.
    \item \textbf{CWE-337: Predictable seed in pseudo-random number generator}: La sicurezza di un generatore \textit{pseudo-casuale} di numeri dipende interamente dall'impredicibilità del seed con cui è stato
    inizializzato. Se un attaccante è in grado di predire il seed che verrà usato per il generatore, allora sarà anche in grado di predire i valori casuali che verranno utilizzati dal programma, rendendolo potenzialmente vulnerabile
    se questi numeri verranno utilizzati per svolgere operazioni sensibili (come, per esempio, la generazione di una chiave crittografica).
\end{itemize}
L'efficacia della strategia di analisi implementata da \textit{Arbiter} è stata verificata utilizzando, per ciascuna classe di vulnerabilità rilevabile, due versioni
dello stesso programma: una vulnerabile e una non vulnerabile.
\subsection{Test effettuato per CWE-131}
Consideriamo il seguente programma vulnerabile a \textit{CWE-131}:
\lstinputlisting[language = C, caption = Un programma in \textit{C} vulnerabile a CWE-131]{../tests/Arbiter/CWE-131-bad.c}
Nel linguaggio \textit{C}, le stringhe sono sempre terminate da un carattere nullo "$\backslash0$". 
Nel programma, l'allocazione del buffer non tiene però conto del carattere terminatore: viene riservato spazio
solamente per i caratteri "effettivi" della stringa, il che causa un buffer overflow al momento della copia. La correzione di questa Vulnerabilità è
triviale: è sufficiente aggiungere $1$ allo spazio allocato per il buffer.
\lstinputlisting[language = C, caption = Versione non vulnerabile del programma presentato nel listing 7.4]{../tests/Arbiter/CWE-131-good.c}
\textit{Arbiter} identifica correttamente la vulnerabilità nella versione vulnerabile del programma. Tuttavia, esso continua a segnalarla anche nella
versione corretta. La rilevazione di questo falso positivo suggerisce la necessità di raffinare i vincoli utilizzati per descrivere
la vulnerabilità nel file VD della CWE-131. 
\begin{table}[H]
    \centering
    \begin{tabular}{|c|c|}
    \hline
    Vulnerability & Result \\ \hline
    CWE-131       & True   \\ \hline
    CWE-134       & False  \\ \hline
    CWE-252       & False  \\ \hline
    CWE-337       & False  \\ \hline
    \end{tabular}
    \caption{Risultati ottenuti da \textit{Arbiter} sul programma riportato nel listing 7.4}
\end{table}
\begin{table}[H]
    \centering
    \begin{tabular}{|c|c|}
    \hline
    Vulnerability & Result \\ \hline
    CWE-131       & True   \\ \hline
    CWE-134       & False  \\ \hline
    CWE-252       & False  \\ \hline
    CWE-337       & False  \\ \hline
    \end{tabular}
    \caption{Risultati ottenuti da \textit{Arbiter} sul programma riportato nel listing 7.5}
\end{table}
\subsection{Test effettuato per CWE-134}
Consideriamo il seguente programma vulnerabile a \textit{CWE-134}:
\lstinputlisting[language = C, caption = Un programma in \textit{C} vulnerabile a CWE-134]{../tests/Arbiter/CWE-134-bad.c}
Nel programma sopra riportato, la vulnerabilità è causata dall'utilizzo diretto della stringa contenuta in \textit{buf}, la quale potrebbe essere stata
manipolata da un attaccante per portare il programma a leggere dati sullo stack o a sovrascrivere la memoria. I risultati presentati da \textit{Arbiter} confermano
la presenza della vulnerabilità:
\begin{table}[H]
    \centering
    \begin{tabular}{|c|c|}
    \hline
    Vulnerability & Result \\ \hline
    CWE-131       & False   \\ \hline
    CWE-134       & True  \\ \hline
    CWE-252       & False  \\ \hline
    CWE-337       & False  \\ \hline
    \end{tabular}
    \caption{Risultati ottenuti da \textit{Arbiter} sul programma riportato nel listing 7.6}
\end{table}
\noindent
Anche in questo caso, la correzione di questa vulnerabilità è triviale: è sufficiente specificare una format string valida che stampi il contenuto del buffer.
\lstinputlisting[language = C, caption = Versione non vulnerabile del programma presentato nel listing 7.6]{../tests/Arbiter/CWE-134-good.c}
A differenza del test precedente, l'analisi effettuata da \textit{Arbiter} non riporta nessun falso positivo durante l'analisi del programma corretto.
\begin{table}[H]
    \centering
    \begin{tabular}{|c|c|}
    \hline
    Vulnerability & Result \\ \hline
    CWE-131       & False   \\ \hline
    CWE-134       & False  \\ \hline
    CWE-252       & False  \\ \hline
    CWE-337       & False  \\ \hline
    \end{tabular}
    \caption{Risultati ottenuti da \textit{Arbiter} sul programma riportato nel listing 7.7}
\end{table}
\subsection{Test effettuato per CWE-252}
Consideriamo il seguente programma vulnerabile a \textit{CWE-252}:
\lstinputlisting[language = C, caption = Un programma in \textit{C} vulnerabile a CWE-252]{../tests/Arbiter/CWE-252-bad.c}
Il mancato controllo sul valore di ritorno di \textit{setuid} porta il programma sopra riportato ad essere potenzialmente
vulnerabile: se infatti l'operazione svolta da \textit{setuid} fallisce, allora il programma potrebbe comunque continuare l'esecuzione
mantenendo i privilegi originari. In un programma più complesso, questo tipo di vulnerabilità potrebbe permettere all'attaccante di eseguire, tramite
il programma, operazioni sensibili di solito riservate ad utenti con privilegi di amministrazione del sistema. Nonostante sia vulnerabile, \textit{Arbiter} non è in grado
di rilevare la presenza della vulnerabilità in questo programma. In particolare, esso non riesce a trovare la \textit{sorgente} da cui far partire la UCSE, portando alla segnalazione
di un falso negativo. È quindi necessario rivedere completamente come sono definite le sorgenti nella VD della CWE-252.
\begin{table}[H]
    \centering
    \begin{tabular}{|c|c|}
    \hline
    Vulnerability & Result \\ \hline
    CWE-131       & False   \\ \hline
    CWE-134       & False  \\ \hline
    CWE-252       & False  \\ \hline
    CWE-337       & False  \\ \hline
    \end{tabular}
    \caption{Risultati ottenuti da \textit{Arbiter} sul programma riportato nel listing 7.8}
\end{table}
\subsection{Test effettuato per CWE-337}
Consideriamo il seguente programma vulnerabile a \textit{CWE-337}:
\lstinputlisting[language = C, caption = Un programma in \textit{C} vulnerabile a CWE-337]{../tests/Arbiter/CWE-337-bad.c}
L'utilizzo del seed \textit{time(NULL)} porta il generatore di numeri pseudo-casuali ad essere inizializzato con il timestamp
corrente in secondi. Se un attaccante riesce a capire approssimativamente quando il seed è stato generato, può provare
a riprodurre \textit{rand()} e trovare il token. L'analisi effettuata da \textit{Arbiter} sul programma è in grado di rilevare correttamente la vulnerabilità:
\begin{table}[H]
    \centering
    \begin{tabular}{|c|c|}
    \hline
    Vulnerability & Result \\ \hline
    CWE-131       & False   \\ \hline
    CWE-134       & False  \\ \hline
    CWE-252       & False  \\ \hline
    CWE-337       & True  \\ \hline
    \end{tabular}
    \caption{Risultati ottenuti da \textit{Arbiter} sul programma riportato nel listing 7.9}
\end{table} \noindent
Per correggere questa vulnerabilità, è necessario utilizzare una catena di generazione di numeri pseudo-casuali meno prevedibile. 
Una buona idea è quella di utilizzare dei valori derivati da sorgenti di "casualità" sicure, come per esempio
il file \textit{/dev/urandom}, il quale contiene numeri casuali generati da un generatore di numeri pseudo-casuali
\textbf{crittograficamente sicuro}.
\lstinputlisting[language = C, caption = Versione non vulnerabile del programma presentato nel listing 7.9]{../tests/Arbiter/CWE-337-good.c}
Nonostante la correzione, \textit{Arbiter} continua a rilevare la vulnerabilità. Risulta quindi necessario raffinare i vincoli che caratterizzano
la presenza di una vulnerabilità nella VD per la CWE-337.
\begin{table}[H]
    \centering
    \begin{tabular}{|c|c|}
    \hline
    Vulnerability & Result \\ \hline
    CWE-131       & False   \\ \hline
    CWE-134       & False  \\ \hline
    CWE-252       & False  \\ \hline
    CWE-337       & True  \\ \hline
    \end{tabular}
    \caption{Risultati ottenuti da \textit{Arbiter} sul programma riportato nel listing 7.10}
\end{table}
\end{document}