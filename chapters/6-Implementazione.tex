\documentclass[../main.tex]{subfiles}
\addbibresource{../bibfile.bib}

\begin{document}

\chapter{Implementazione}
In questo capitolo, si fornirà una descrizione dettagliata dell'architettura della piattaforma Binoculars.
In particolare, verranno illustrare le implementazioni sia del frontend che del backend,
illustrando come la loro collaborazione permette all'utente di usare semplicemente le funzionalità offerte dalla piattaforma.
\section{Architettura}
L'architettura della piattaforma si suddivide in due componenti principali:
\begin{itemize}
    \item \textbf{Frontend}: Il frontend dell'applicazione comprende la logica di caricamento dei file binari e la presentazione
    dei risultati delle analisi. Questa componente è stata sviluppata utilizzando \textbf{SvelteKit}, un framework leggero e robusto per lo sviluppo
    di applicazioni web.
    \item \textbf{Backend}: Il backend dell'applicazione consiste in un server web sviluppato utilizzando il framework python \textbf{Flask}.
    In particolare, il server si occupa delle seguenti mansioni
    \begin{itemize}
        \item Gestione dei file binari caricati dall'utente
        \item Gestione della sessione HTTP
        \item Decompiling e Disassembly del file binario
        \item Applicazione delle diverse tecniche di analisi disponibili e generazione della risposta in formato \textit{JSON}
    \end{itemize}
\end{itemize}
\section{Backend}


















\end{document}