\documentclass[../main.tex]{subfiles}
\addbibresource{../bibfile.bib}

\begin{document}


\chapter{Stato dell'arte}
\markboth{Capitolo 2}{}
\label{chap:related}
Questo capitolo tratta una rassegna di alcune metodologie, tecniche e soluzioni attualmente disponibili per risolvere il problema della ricerca automatica di vulnerabilità in file binari.
Verranno in particolare approfonditi alcuni approcci basati su analisi statica, analisi dinamica e su tecniche di apprendimento automatico.
Per ciascuna metodologia presentata, verranno dettagliati il suo funzionamento generale, le sue capacità di analisi e le sue limitazioni

\section{Metodologie basate su tecniche di analisi statica}
L'analisi statica di un programma consiste in un'insieme di metodologie, tool e algoritmi che permettono l'analisi del codice sorgente o della
sua rappresentazione binaria (per esempio, un file eseguibile) senza che il programma venga effettivamente eseguito \cite{static_analysis_introduction}.
Questa tecnica è ampiamente adottata nell'ambito della ricerca delle vulnerabilità, in quanto consente di inferire e determinare se
certe proprietà sono soddisfatte (per esempio, le condizioni che possono portare ad una certa vulnerabilità) senza direttamente eseguire il programma.
Tuttavia, l'analisi statica condotta direttamente su un file binario è intrinsecamente più complessa rispetto all'analisi statica del codice sorgente:
le principali difficoltà risiedono nella mancanza di informazioni riguardante i tipi e la struttura ad alto livello del codice \cite{Review_of_static_analysis} e nella necessità di gestire e rappresentare adeguatamente le operazioni riguardanti la memoria \cite{CodeSurfer}.
Nonostante queste sfide, nel corso degli anni sono stati sviluppati diversi approcci e metodologie di analisi statica progettati per effettuare la ricerca di vulnerabilità all'interno di file binari.
Queste tecniche, tuttavia, possono produrre un elevato numero di falsi positivi e falsi negativi : poiché non effettuano un'esecuzione concreta del programma,
esse devono effettuare diverse assunzioni sul suo stato a runtime. Ciò potrebbe quindi portare i tool basati su questa tipologia di analisi a segnalare vulnerabilità in porzioni di programma non vulnerabili.
\subsection{Taint analysis statica: Bintaint}
La \textit{taint analysis} (o \textit{taint checking}) è una tecnica di analisi che mira a tracciare e monitorare la propagazione di flussi di dati inaffidabili o 
potenzialmente dannosi all'interno del programma. La taint analysis si compone di tre elementi chiave:
\begin{enumerate}
    \item \textbf{Sorgenti} (Sources): Sono i punti del programma dove si origina un flusso di dati inaffidabile. Una sorgente potrebbe per esempio essere l'input di un utente oppure i dati
    letti da un file.
    \item \textbf{Propagazione}: Viene effettuato un monitoraggio continuo della propagazione nel programma dei dati provenienti da una sorgente
    \item \textbf{Sink}: Sono i punti del programma che effettuano operazioni sensibili, come per esempio l'accesso al filesystem o la chiamata ad operazioni di libreria non sicure. 
\end{enumerate}
Una possibile vulnerabilità verrà quindi rilevata quando il programma permette ad un dato "tainted" di raggiungere un sink; ciò può avvenire quando, per esempio, il dato
non viene adeguatamente sanificato.
\newline
\textbf{Bintaint} è un tool di parsing capace di effettuare taint analysis statica su file binari  \cite{Bintaint}.
Il taint analyzer proposto è basato sul tool commerciale di reverse engineering \textit{IDA}, il quale viene utilizzato per recuperare il codice assembly dal codice binario, 
ed è implementato utilizzando il linguaggio funzionale \textit{OCaml}.
Bintaint è composto da quattro moduli distinti:
\begin{itemize}
    \item \textbf{Decoder module}: Questo modulo si occupa di tradurre il codice assembly recuperato da IDA in una rappresentazione in un linguaggio intermedio chiamato \textit{REIL} (Reverse Engineering Intermediate Language), 
    le quali espressioni verranno a loro volta convertite in espressioni simboliche.
    \item \textbf{Taint Processing Configuration Module}: Questo modulo gestisce la configurazione per l'inizializzazione della taint analysis, leggendo la configurazione fornita dall'utente in formato XML, la quale
    dovrà contenere tutte le informazioni necessarie per effettuare la taint analysis. Questo modulo si occupa inoltre di stabilire una relazione tra l'input esterno e le varie sorgenti definite
    \item \textbf{Expression Parsing Module}: Questo modulo si occupa di definire come avviene la propagazione dei flussi di dati tainted all'interno del programma
    \item \textbf{TCFG Generation Module}: Questo modulo si occupa di generare una struttura a grafo diretta chiamata \textit{Taint Control Flow Graph}, la quale rappresenterà tutte le possibili aree del programma che un determinato flusso tainted può raggiungere.
    L'analisi del TCFG permetterà quindi di evincere se un determinato sink dipende dai dati generati da una determinata sorgente.
\end{itemize}
L'approccio proposto dal tool permette di ridurre il numero di falsi positivi e falsi negativi rilevati rispetto ad una tain analysis tradizionale; inoltre, l'utilizzo del linguaggio intermedio REIL
permette al tool di essere facilmente integrabile in sistemi di analisi più complessi, a patto che anch'essi utilizzino lo stesso linguaggio di rappresentazione intermedia. 
Tuttavia il tool risulta comunque dipendente dall'input dell'analista; l'accuratezza dell'analisi dipenderà quindi dalla corretta definizione di sorgenti, sink e propagazione da parte
di quest'ultimo. Infine, Bintaint si basa sul framework commerciale IDA, il quale non offre tutte le sue funzionalità nella sua versione gratuita.
\subsection{Binary Code Similarity Detection: VulneraBin}
Quando l'obbiettivo dell'analisi è ricercare una vulnerabilità nota (per esempio, una debolezza già documentata), è possibile adottare una strategia chiamata \textit{Binary Code Similarity Detection} (BCSD).
Questo approccio si basa sul confronto il codice binario del programma in esame con la firma (il codice binario) della vulnerabilità.
Se l'algoritmo di analisi rileva segmenti di codice con un elevato grado di somiglianza con la firma della vulnerabilità, allora è altamente probabile che il programma
contenga quella vulnerabilità. Un algoritmo di decisione determinerà se il programma contiene effettivamente la vulnerabilità.
Tuttavia, poiché il codice contente la vulnerabilità spesso richiede solo piccole modifiche per fare in modo che esso non sia più vulnerabile (l'aggiunta di un controllo, l'impostazione di permessi aggiuntivi, ...), il codice binario del programma corretto e il programma originale
saranno molto simili; potenzialmente portando l'analizzatore a segnalare dei falsi positivi \cite{Survey_of_Binary_Code_Security_Analysis}. \newline
\textit{VulneraBin} \cite{VulneraBin} è un tool che effettua un'analisi BCSD attraverso una metrica di similarità basata su hashing, strutturando il processo nelle seguenti fasi:
\begin{enumerate}
    \item \textbf{Re-ottimizzazione del linguaggio di rappresentazione intermedia (IR)}: Il codice assembly viene dapprima tradotto nel linguaggio di rappresentazione intermedia VEX-IR, il quale utilizzo
    mira ad appiattire le eventuali differenze sintattiche derivanti dall'utilizzo di registri diversi, istruzioni diverse per l'assegnamento o metodologie di ottimizzazione introdotte dai vari compilatori.
    Successivamente, viene applicata un'ulteriore ottimizzazione sul codice intermedio per eliminare le differenze residue che potrebbero ancora persistere a causa delle diverse tecniche di ottimizzazione dei compilatori.
    \item \textbf{Program Slicing}: Il program slicing è una tecnica di anali statica che, partendo da un sottoinsieme dei comportamenti di un programma, ne produce una versione minimale, chiamata "slice",  la quale mantiene esattamente lo stesso sottoinsieme di comportamenti.
    Poiché questa tecnica è alla base delle analisi offerte dalla piattaforma, verrà ulteriormente approfondita nel \textit{Capitolo 3} di questa tesi.
    \item \textbf{Strand normalization}: Una "strand" è definita come l'insieme di istruzioni contigue richieste per computare il valore di una specifica variabile \cite{Statistical_similarities_in_binaries}.
    Gli strand vengono normalizzati rinominando i registri utilizzati durante le varie operazioni, andando così ad eliminare eventuali differenze sintattiche introdotte dai compilatori
    \item \textbf{Similarity evaluation}: Viene effettuato un confronto fra gli hash MD5 calcolati sugli strand normalizzati e gli hash delle vulnerabilità contenute in un database. Se la similarità supera una certa soglia definita manualmente, allora
    il binario sarà considerato vulnerabile.
\end{enumerate}
Nonostante l'approccio proposto porti ad un miglioramento della complessità computazionale dell'analisi e alla mitigazione del numero di falsi positivi e negativi rilevati, l'affidabilità dell'analisi rimane comunque legata ad una soglia scelta manualmente dall'analista.
Sarà quindi necessario che quest'ultimo imposti una soglia ottimale per ogni specifico binario o classe di vulnerabilità, compromettendo quindi l'automazione del processo.
\section{Metodologie basate su tecniche di analisi dinamica}
L'analisi dinamica consiste nell'osservazione del comportamento di un programma mentre esse viene eseguito in un determinato ambiente d'esecuzione.
Per consentire questo tipo di analisi, i tool che implementano questo tipo di tecniche devono effettuare un processo chiamato \textit{instrumentation}, il quale
consiste nell'aggiungere codice di analisi all'interno del programma da analizzare in modo tale che venga eseguito insieme a quest'ultimo senza modificarne il normale
flusso di esecuzione \cite{DBA_DBI}. I risultati ottenuti tramite l'analisi dinamica sono generalmente più precisi rispetto ai risultati ottenuti effettuando un'analisi statica
del programma, poiché non vi è più la necessità di effettuare un'astrazione riguardo i valori computati o il cammino intrapreso dal programma sotto analisi.
Tuttavia, poiché l'esecuzione concreta di un programma richiede la scelta di un insieme di input concreti con il quale eseguirlo, i risultati ottenuti tramite queste tecniche non sono
generalizzabili, in quanto l'insieme di input scelto potrebbe non essere rappresentativo di tutti i possibili cammini d'esecuzione del programma \cite{Synergy_Duality}.
\subsection{Fuzzing: }
Il fuzzing è una tecnica di analisi che consiste nell'osservare il comportamento del programma quando esso
riceve degli input casuali o malformati. Se a causa di questi input avviene un blocco dell'esecuzione oppure un crash, allora il programma
potrebbe contenere una debolezza software oppure un problema di implementazione. Generalmente, i principali componenti del fuzzing sono  \cite{Fuzzing_soa}:
\begin{itemize}
    \item \textbf{Programma obbiettivo}: Il programma da analizzare, il quale può essere rappresentato sia dal suo codice binario sia dal suo codice sorgente. Poiché l'accesso a quest'ultimo è a volte ostico in situazioni reali, i software di fuzzing spesso
    hanno come programma obbiettivo il solo codice binario.
    \item \textbf{Monitor}: Raccoglie informazioni riguardanti l'esecuzione del programma.
    \item \textbf{Input generator}: Si occupa della generazione degli input, la quale può avvenire in due modi distinti:
    \begin{itemize}
        \item \textbf{Grammar-based}: Gli input vengono generati utilizzando una grammatica 
        \item \textbf{Mutation-based}: Gli input vengono generati usando dei file seed, i quali vengono mutati casualmente oppure utilizzando
        delle strategie di mutazione ben definite.
    \end{itemize}
    \item \textbf{Bug detector}: Quando il programma va in crash o riporta degli errori, questo modulo recupera e analizza le informazioni rilevanti per determinare se vi è la presenza di un "bug" (una debolezza, una vulnerabilità, ...).
    \item \textbf{Bug filter}: Non tutti i "bug" sono effettivamente delle vulnerabilità; è quindi necessaria un'operazione di filtraggio per scartare tutte quelle problematiche che non risultano sfruttabili da un attaccante.
\end{itemize}









\end{document}