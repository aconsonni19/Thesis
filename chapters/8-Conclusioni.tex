\documentclass[../main.tex]{subfiles}
\addbibresource{../bibfile.bib}

\begin{document}

\chapter{Conclusioni}
In un mondo sempre più digitalizzato, nessun processo di sviluppo software si può esimere dall'integrare al suo interno tematiche di sicurezza.
Effettuare un'analisi di sicurezza accurata contribuisce non solo alla qualità del software rilasciato, ma anche alla protezione di
dati sensibili dei suoi utenti. È quindi imperativo che la sicurezza di un'applicazione sia considerata \textbf{sin dalle fasi iniziali della sua progettazione}.
Tuttavia, questo processo ci complica notevolmente quando l'analista ha accesso al solo file eseguibile e deve ricorre al \textit{reverse engineering}, il quale richiede
competenze specialistiche. Questa difficoltà viene risulta ulteriormente accentuata quando l'esperienza dell'analista in questo campo è limitata.
\textit{Binoculars} è stata progettata seguendo due principi cardine: la \textbf{semplicità} di utilizzo e la \textbf{democratizzazione} delle tecniche di ricerca automatica delle vulnerabilità. 
Il livello di astrazione che viene offerto dall'interfaccia della piattaforma permette anche ad analisti poco esperti di effettuare una prima analisi esplorativa del file binario.
Inoltre, la piattaforma permette di utilizzare due strategia di ricerca automatica delle vulnerabilità, \textit{Arbiter} e \textit{VulnDetect}, per effettuare un'identificazione
rapida di alcune delle vulnerabilità più comuni che il programma potrebbe contenere. Nonostante \textit{Binoculars} non possa sostituirsi a strumenti di reverse engineering più avanzati, come per esempio
\textit{gdb}, la piattaforma risulta una soluzione \textbf{pratica e a basso costo} per facilitare l'integrazione delle pratiche di sicurezza all'interno dei processi di sviluppo software, migliorandone al contempo
la velocità di applicazione.
\section{Sviluppi futuri}
\textit{Binoculars} è stata progetta per essere facilmente estensibile con ulteriori raffinamenti e nuove tecniche di analisi. Alcuni sviluppi futuri di questo progetto sono:
\begin{itemize}
    \item Permettere il caricamento e l'analisi di file binari con formati eseguibili diversi da \textit{ELF}.
    \item Integrare ulteriori tecniche di analisi, permettendo quindi alla piattaforma di coprire la ricerca di più classi di vulnerabilità.
    \item Effettuare una sperimentazione più approfondita delle tecniche di analisi attualmente rese disponibili dalla piattaforma, utilizzando un dataset più strutturato e ricco di esempi.
    \item Raffinare ulteriormente le \textit{Vulnerability Description}, così che essa possano fornire una caratterizzazione più precisa della vulnerabilità a cui fanno riferimento.
\end{itemize}

\end{document}